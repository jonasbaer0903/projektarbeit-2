\documentclass[a4paper, 12pt, oneside, openany]{scrbook}

\usepackage[backend=biber,style=alphabetic,citestyle=authoryear ]{biblatex}
\usepackage[ngerman]{babel}
\usepackage[T1]{fontenc}
\usepackage[utf8]{inputenc}
\usepackage[]{hyperref}
\usepackage{graphicx}
\usepackage{epstopdf}
\usepackage{float}
\usepackage{acronym}
\usepackage{booktabs}
\usepackage{caption}
\usepackage{csquotes}
\usepackage{fancyhdr}
\usepackage{url}
\usepackage{listings}
\usepackage{blindtext}
\usepackage{courier}
\usepackage{comment}
\usepackage{pdfpages}
\usepackage{subcaption} 

\newcommand*{\todo}[1]{\colorbox{yellow}{\parbox{1\linewidth} {TODO: #1}}} % Yellow TODO box u.A. https://tex.stackexchange.com/a/319000/129441

\renewcommand*{\headfont}{\normalfont}
%\renewcommand*{\multicitedelim}{\addsemicolon\space}
\renewcommand*{\headrulewidth}{0pt}
\renewcommand*{\arraystretch}{1.5}
\newcommand{\code}{\texttt} % code in monospace font, wont hyphenate https://tex.stackexchange.com/a/36031/129441

\newcommand{\compcite}{\parencite[vgl.][]}

\setlength{\parskip}{1.5ex}

\MakeOuterQuote{"}

\hypersetup{
  colorlinks=true,
  linkcolor=black,
  citecolor=blue,
  urlcolor=cyan
}

% Damit refs automatisch verlinkt werden und "Abb. 1" generiert wird
\usepackage[ngerman]{cleveref}

% "Tabelle 1" wird z.B. zu "Tab. 1" ge�ndert
\addto\captionsngerman{
    \crefname{chapter}{kap.}{kap.}
    \Crefname{chapter}{Kap.}{Kap.}
    \crefname{subsubsection}{abs.}{abs.}
    \Crefname{subsubsection}{Abs.}{Abs.}
    \crefname{subsection}{abs.}{abs.}
    \Crefname{subsection}{Abs.}{Abs.}
    \crefname{section}{abs.}{abs.}
    \Crefname{section}{Abs.}{Abs.}
    \crefname{figure}{abb.}{abb.}
    \Crefname{figure}{Abb.}{Abb.}
    \crefname{table}{tab.}{tab.}
    \Crefname{table}{Tab.}{Tab.}
    \renewcommand{\figurename}{Abb.}
  	\renewcommand{\tablename}{Tab.}
}

\begin{document}

\mainmatter

\chapter{Kapitel}

\begin{figure}[H]
    \centering
    \includegraphics[width=0.15\textwidth]{images/kis.png}
    \caption{Kaufland Logo.}
    \small Das derzeitige Logo von Kaufland.
    \label{fig-kis-small}
\end{figure}

\begin{table}[H]
    \centering
    \begin{tabular}{l c p{5cm}}
        \toprule
        Überschrift 1 & Überschrift 2 & Beschreibung \\
        \midrule
        Eins & Zwei & Dies ist ein sehr langer Text der in die nächste Zeile rut-schen soll. &  \\
        Vier & Fünf & Falls dies nicht passiert, ist die Tabelle falsch konfigu-riert. \\
        \bottomrule
    \end{tabular}
    \caption{Dies ist eine Beschreibung.}
    \label{tab-test}
\end{table}
Hier seht ihr \Cref{fig-kis-small} und \Cref{tab-test}.

\nocite{*}

\backmatter

\sloppy

\cleardoublepage
\phantomsection

\end{document}
