\chapter{Figures (Illustrationen)}

\begin{figure}[H]
    \centering
    \includegraphics[width=0.25\textwidth]{images/kis}
    \caption{Kaufland Logo.}
    \small Das derzeitige Logo von Kaufland.
    \label{fig-kis-big}
\end{figure}

\vspace{15pt} % Vertikaler Abstand

\begin{figure}[H]
    %\centering
    \begin{subfigure}{.45\textwidth}
      \centering
      \includegraphics[width=.325\linewidth]{images/kis}
      \caption{Kaufland Logo.}
      \label{fig-kis}
    \end{subfigure}%
    \hspace{1em} % Horizontaler Abstand
    \begin{subfigure}{.45\textwidth}
      \centering
      \includegraphics[width=.9\linewidth]{images/sitios}
      \caption{SITIOS Logo.}
      \label{fig-sitios}
    \end{subfigure}

    \centering
    \caption{Mehrere Figures nebeneinander.}
    \small Illustrationen die zusammenhängen können nebeneinander mit jeweils einer kurzen Beschreibung dargestellt werden. Diese werden in diesem Text referenziert. \newline (a) ist Rot. (b) ist Grau. % \newline erzwingt Zeilenumbruch
    \label{fig-search}
\end{figure}